\documentclass{article}
\usepackage{graphicx}
\usepackage{float}

\begin{document}
	
	\title{Dispositivos de red Hubs}
	\author{Lautaro Ricci}
	\date{\today}
	
	\maketitle
	
	\section{Conexión de Dispositivos}
	
	Realizar la interconexión entre dos PC's podria ser algo innecesario, por lo que para una mayor utilidad realizamos la conexión a traves de un cable cruzado entre la PC y un Server (arquitectura conocida como cliente-servidor).
	
	Con el comando \textbf{'\textbf{ipconfig /all}'} podemos examinar tanto las direcciones físicas (IP) y lógicas (MAC) tanto del PC1, como del Server.
	\\	
	
	\textbf{Direcciones Físicas}
	\begin{itemize}
		\item \textbf{PC:} 000C.CF40.25C8
		\item \textbf{Server:} 000C.CF40.25C8
	\end{itemize}

	\textbf{Direcciones Lógicas:}
	\begin{itemize}
		\item \textbf{PC:} 192.168.0.1
		\item \textbf{Server:} 192.168.0.100
	\end{itemize}
	
	
	
	\section{Extendiendo la Red}
	\begin{table}[h!]
		\centering{Direcciones MAC e IP de los Dispositivos}
		\begin{tabular}{|l|l|l|}
			\hline
			\textbf{Dispositivo} & \textbf{Dirección MAC} & \textbf{Dirección IP} \\
			\hline
			PC1      & 000C.CF40.25C8 & 192.168.0.1 \\		
			\hline
			PC2      & 0010.1183.9B19 & 192.168.0.3 \\
			\hline
			PC3      & 00D0.FF57.95EA & 192.168.0.2 \\
			\hline
			Server  & 0002.17D4.871E & 192.168.0.100 \\
			\hline			
		\end{tabular}
	\end{table}
	
	No es necesario realizar el \textbf{'ipconfig /all'} a cada dispositivo, sino que enviando los PDU al server y luego el utilizando el comando \textbf{'arp -a'} podremos ver todas las direcciónes IP y direcciónes MAC de los diferentes dispositivos.

 \textbf{Imagen Demostrativa:}
	\begin{figure}
	    \centering
	    \includegraphics[width=0.5\linewidth]{WhatsApp Image 2024-08-30 at 13.52.43_c6d8f11e.jpg}
	\end{figure}
	
	\section{Dominios de colisión}
	
		En la configuración con 3 hubs centrales y los dispositivos conectados a estos hubs, el número total de dominios de colisión es 3. Cada hub central crea su propio dominio de colisión, y la interconexión entre hubs permite la comunicación entre los dominios, pero no cambia el número total de dominios de colisión.
\\
\\
\\
  \textbf{Link al git:} https://github.com/LautaroRicci/Dispositivos-de-red-Hubs
			

\end{document}

